%%%%%%%%%%%%%%%%%%%%%%%%%%%%%%%%%%%%%%%%%%%%%%%%%%%%%%%%%%%%%%%
%
% Welcome to Overleaf --- just edit your LaTeX on the left,
% and we'll compile it for you on the right. If you open the
% 'Share' menu, you can invite other users to edit at the same
% time. See www.overleaf.com/learn for more info. Enjoy!
%
%%%%%%%%%%%%%%%%%%%%%%%%%%%%%%%%%%%%%%%%%%%%%%%%%%%%%%%%%%%%%%%
\documentclass[11pt,titlepage]{article}		% The percent symbol in your code starts a comment.  The comment ends at the next linebreak.

\usepackage[english]{babel} 		% Packages add functionality and style conventions to your documents. Don't edit this section!
\usepackage{fullpage}				% Eliminates wasted space
\usepackage[utf8]{inputenc}			% Necessary for character encoding
\usepackage{amsmath, amssymb,amsthm}% Required math packages
\usepackage{graphicx}				% For handling graphics
\usepackage[colorinlistoftodos]{todonotes}	% For the fancy "todo" stuff
\usepackage{hyperref}				% For clickable links in the final PDF
%\usepackage{titling}				% To take less space at the top of the page with the title
%\setlength{\droptitle}{-2cm}
%\pretitle{\Large\scshape}%{\begin{flushright}\Large\scshape}
%\posttitle{\par\end{flushright}}
%\preauthor{\large\scshape}
%\postauthor{\par\end{flushright}}
%\predate{\large\scshape}
%\postdate{\par\end{flushright}}
\linespread{1.5}

\newcommand{\set}[1]{\left\{ {#1} \right\}}
\newcommand{\setof}[2]{{\left\{#1\,\colon\,#2\right\}}}


% Type `\C' for the complex numbers, `\H' for the quarternions, etc.
\def\C{{\mathbb C}}
\def\H{{\mathbb H}}
\def\Z{{\mathbb Z}}
\def\Q{{\mathbb Q}}
\def\R{{\mathbb R}}
\def\N{{\mathbb N}}


%\Alpha{homeworkresults}

\newtheorem{theorem}{Theorem}
\renewcommand*{\thetheorem}{\Roman{theorem}}
%\setcounter{theorem}{2}
\newtheorem{lemma}[theorem]{Lemma}
\newtheorem{prop}[theorem]{Proposition}
\newtheorem{claim}[theorem]{Claim}
\newtheorem{example}[theorem]{Example}
\newtheorem{conjecture}[theorem]{Conjecture}
\theoremstyle{definition}
\newtheorem{definition}{Definition}
\theoremstyle{theorem}




\title{\sc Math 212 Final Exam}

\author{Kenyon Geetings}

\date{Draft date: \today}

\begin{document}

\maketitle



\clearpage

\begin{conjecture}
	Use methods from class to find a closed form for the sequence
	\begin{align*}
    4,4,10,28,64,124,214,340,508,724,...
    \end{align*}
\end{conjecture}

\begin{proof}
To find a closed form for the sequence, we can first find what degree we get a common difference. If we assume the first term is $a_0$, we can create a chart.

\begin{displaymath}
\begin{array}{c | c | c | c}
a_{n} & \Delta _{1} & \Delta _{2} & \Delta _{3} \\
\hline
4 & 0 & 6 & 6\\
4 & 6 & 12 & 6\\
10 & 18 & 18 & 6\\
28 & 36 & 24 & 6\\
64 & 60 & 30 & 6\\
124 &90 & 36 & 6\\
214 & 126 & 42 & 6\\
340 & 168 & 48 & -\\
508 & 216 & -& -\\
724 & - &-  &- 
\end{array}
\end{displaymath}
From the table, we can see this is a degree-3 polynomial, so it will follow the form $a_n = an^3 + bn^2 + cn + d$. To solve, we plug in known values of $a_n$ and solve the system of equations.

\begin{align*}
a_n &= an^3 + bn^2 + cn + d\\
a_0 = 4 &= a\cdot0 + b\cdot0 + c\cdot0 + d\\
\end{align*}
From this equation we see that $d=4$.
\begin{align*}
a_1 = 4 &= a + b + c + 4\\
a_2 = 10 &= 8a + 4b + 2c + 4\\
a_3 = 28 &= 27a + 9b + 3c + 4\\
\end{align*}
After solving the system of equations, we find that $a=1,b=0,c=-1,d=4$. Thus, our closed form equation is $a_n = n^3 -n + 4$.
\end{proof}





\clearpage
\begin{conjecture}
Let $A_1,A_2,...,A_n$ be sets where $n \ge 2$. Suppose for any two sets $A_i$ and $A_j$ that either $A_i \subseteq A_j$ or $A_j \subseteq A_i$. Prove by induction that one of these $n$ sets is a subset of all of them.
\end{conjecture}
\begin{proof}
By Induction. Let $P(n)$ be the statement for $A_1,A_2,...,A_n$, for two sets $A_i,A_j$, $A_i \subseteq A_j$ or $A_j \subseteq A_i$.

Base Case: When $n = 2$, we have the sets $A_1, A_2$. Thus, either $A_1 \subseteq A_2$ or $A_2 \subseteq A_1$. Therefore, one of these sets is a subset of all other sets.

Inductive Hypothesis: Assume $P(k)$ is true for all $k \ge 2$, which means we have
\begin{align*}
A_1,A_2,...,A_k.
\end{align*}

Inductive Step: We can then consider $P(k+1)$ which results in $A_{k+1}$
\begin{align*}
A_1,A_2,...,A_k,A_{k+1}.
\end{align*}
By our base case, we know that for the first $k$ elements, one of them will be the subset of all other sets, which we can call set $A_a$. Thus, we are left with 2 sets: $A_a$ and $A_{k+1}$. By our definition, one of these sets must be the subset of the other, meaning that $A_a$ is subset to all other sets, or $A_{k+1}$ is subset to all other sets.

Therefore, since $P(k)$ is true for all $k \ge 2$, we see that $P(n)$ is true for all $n \ge 2$.

\end{proof}






\clearpage
\begin{conjecture}
    Let $T$ be a tree with at least two vertices and let $v$ be a vertex of $T$. Let $T \setminus v$ denote the graph obtained from $T$ by deleting $v$ and any edges incident to $v$. If $T \setminus v$ is a tree, prove that $v$ is a leaf.
\end{conjecture}
\begin{proof}
Assume that we have a graph that is a tree with at least two vertices $a,b$. Then between those vertices $a,b$ there is only one unique path, given by Proposition 13.2.1. This also means that if we remove a vertex that has a degree greater than 1, then we will also remove at least two edges, which means we remove a path between at least one pair of vertices. Thus, if $T$ is a tree, and $T\setminus v$ must be a tree, then $v$ must not break any paths, since doing so would make $T\setminus v$ a forest. Therefore, since the loss of $v$ must not remove any other vertices from the tree, then $v$ must have only one incident edge, meaning that it is a leaf, given by Proposition 13.2.3.
\end{proof}

\end{document}